\chapter{Conclusion and Future Work}


\section{Conclusion}

This thesis presented Gelato-30B-A3B, a state-of-the-art grounding model for GUI computer-use tasks. Through careful data curation, novel filtering approaches, and effective reinforcement learning training, we achieved significant improvements over prior work on multiple benchmarks.

Our key contributions include:

\begin{itemize}
    \item \textbf{Click-100k}: A high-quality, open-source dataset built through principled filtering and enrichment
    \item \textbf{Filtering methodology}: Model-based difficulty and alignment filtering that significantly improves dataset quality
    \item \textbf{Training recipe}: Effective RL training approach building on GRPO with practical simplifications
    \item \textbf{Strong results}: State-of-the-art performance on ScreenSpot-Pro (63.88\%) and OS-World-G (69.15\% / 74.65\%)
    \item \textbf{Agent performance}: Demonstrated end-to-end agent improvements on OS-World (61.85\% with human evaluation)
\end{itemize}

Beyond these technical contributions, our work highlights important challenges in agent evaluation and reproducibility. The ~3 percentage point gap between automated and human evaluation, combined with non-deterministic planning models and incomplete evaluation functions, suggests that the field needs more rigorous evaluation practices.

Looking forward, grounding models remain a critical component of computer-use agents. As we push toward more capable and general-purpose agents, continued improvements in grounding accuracy, efficiency, and robustness will be essential. We hope that our open-source release of Click-100k, trained models, and evaluation artifacts will accelerate progress in this important area.

The path to truly general-purpose computer-use agents is long, but strong grounding models like Gelato represent an important step forward. By combining high-quality data, effective training methods, and rigorous evaluation, we can continue to narrow the gap between human and machine capabilities in navigating digital interfaces.

\chapter{Future Work}