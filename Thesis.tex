\documentclass[a4paper, oneside]{discothesis}

\usepackage[utf8]{inputenc}
\usepackage[T1]{fontenc}
\usepackage{bookmark}
\usepackage{xcolor}
\usepackage{listings}
\usepackage{enumitem}
\usepackage{subcaption}

%%%%%%%%%%%%%%%%%%%%%%%%%%%%%%%%%%%%%%%%%%%%%%%%%%%%%%%%%%%%%%%%%%%%%%%%%%%%%%%%%%%%%%%%%%%%%%%%%
% DOCUMENT METADATA

\thesistype{Master's Thesis} % Master's Thesis, Bachelor's Thesis, Semester Thesis, Group Project
\title{Gelato-30B-A3B: Training a State of the Art Model Grounding Model}
\author{Aylin Akkus}
\email{aakkus@ethz.ch}
\institute{Distributed Computing Group \\[2pt] Computer Engineering and Networks Laboratory \\[2pt] ETH Zürich}
\supervisors{Prof.\ Dr.\ Roger Wattenhofer, Prof.\ Dr.\ Ludwig Schmidt}
\date{\today}

%%%%%%%%%%%%%%%%%%%%%%%%%%%%%%%%%%%%%%%%%%%%%%%%%%%%%%%%%%%%%%%%%%%%%%%%%%%%%%%%%%%%%%%%%%%%%%%%%

\begin{document}

\frontmatter % do not remove this line
\maketitle

\cleardoublepage

\begin{acknowledgements}
I would like to express my sincere gratitude to my supervisors, Prof.\ Dr.\ Roger Wattenhofer, Prof.\ Dr.\ Ludwig Schmidt, and Prof.\ Dr.\ Yejin Choi, for their invaluable guidance and support throughout my Master's thesis. Their expertise and insightful feedback have been instrumental in shaping this research.

I am also grateful to Anas Awadalla and Drubha Ghosh for their collaboration on the project. The discussions and support within the team have been a great source of motivation.
I furthermore thank Florian Brand for sharing details on OS-World task quality issues and Yan Yang for details on GTA1 agent evaluation.

I gratefully acknowledge computing time granted for the project synthesis by the JARA on the supercomputer JURECA at Jülich Supercomputing Center (JSC), as well as storage resources on JUST granted and operated by JSC and supported by the Helmholtz Data Federation (HDF).

Finally, I am deeply thankful to my life partner, Mert Unsal, for his constant love, patience, and understanding throughout this journey.

\paragraph{Unique Contributions.}
As most of this work was possible only as a collaborative effort, I would like to outline my unique contributions to this Master's thesis. In particular, I:
\begin{itemize}
    \item Participated in the search and normalization of the existing data sources.
    \item Ran experiments with different filtering methods for the difficulty-based filtering, trained models, and evaluated the results on the offline benchmarks.
    \item Developed methods to utilize APIs (DOM trees and Accessibility Trees) to supplement training data beyond the web, leveraging richer semantic information to generate instructions.
    \item Studied the effect of dense vs.\ sparse rewards on model performance during reinforcement learning training.
    \item Wrote an agentic harness to evaluate the model on the OS-World benchmark and conducted human evaluation to mitigate the limitations of automated evaluators.
    \item Wrote this thesis in full (all text, figure selection and placement, and experimental narration).
\end{itemize}

\paragraph{Scope and Non-Contributions.}
\begin{itemize}
    \item I did not participate in the video annotation process for the supplemental data collection.
    \item I did not participate in running the online OS-World evaluations using containerization and VMs.
\end{itemize}
\end{acknowledgements}


\begin{abstract}
Recent progress in machine learning has been driven largely by scaling compute, model size, and training data---yet of these three pillars, training data has received the least systematic attention. This gap is especially pronounced for Computer-Use agents, where limited heterogeneous data sources exist and data practices remain closed source.

In this thesis we apply a data-centric methodology to the Graphical User Interface (GUI) grounding problem and introduce \textbf{Click-100k}, an open-source dataset assembled through a complete pipeline of data curation, filtering, and quality control across diverse GUI domains. We then train \textbf{Gelato-30B-A3B}, a state-of-the-art grounding model for GUI computer-use tasks, on Click-100k using reinforcement learning. Gelato achieves 63.88\% accuracy on ScreenSpot-Pro and 69.15\,/\,74.65\% on OS-World-G\,/\,OS-World-G (Refined), surpassing prior specialized grounding models such as GTA1-32B and much larger vision-language models including Qwen3-VL-235B-A22B-Instruct. When paired with GPT-5 as the planning backbone, Gelato enables strong agentic performance at 58.71\% automated success rate (61.85\% with human evaluation) versus 56.97\% (59.47\% with human evaluation) for GTA1-32B on OS-World. We describe the complete pipeline---from data curation and filtering to reinforcement learning training---that enables these results and open-source both the dataset and the model.
\end{abstract}

\tableofcontents

\mainmatter % do not remove this line

% Main Chapters (5 chapters as per thesis structure)
% Chapter 1: Introduction (including Related Work)
\chapter{Introduction}
\section{Motivation}
Recent progress in machine learning has been largely driven by scaling laws~\cite{kaplan2020scaling,hestness2017deeplearningscalingpredictable,aghajanyan2023scalinglawsgenerativemixedmodal,cherti2023reproducible}, as demonstrated by models such as GPT-4, Stable Diffusion, and CLIP. These advances rest on three pillars: (1) compute, (2) model size, and (3) training data. Of these, training data has received the least systematic attention—despite its crucial role, datasets themselves are rarely the subject of active research~\cite{sambasivan2021dataset}. On the model side, experimentation is relatively straightforward: given sufficient compute, permutations of width, depth, normalization, and training hyperparameters can be rigorously evaluated, yielding consistent improvements over the years (Touvron et al., 2023a,b; Elsen et al., 2023). The dataset side, however, is murkier. Most large-scale training sets are not publicly released, leaving the community to attempt open reproductions (Schuhmann et al., 2021, 2022; Byeon et al., 2022; Gao et al., 2020)—efforts that are often one-off and lack the iterative refinement that models enjoy. While recent initiatives such as DataPerf, DataComp, and MetaCLIP (Mazumder et al., 2022; Gadre et al., 2023; Xu et al., 2023) have begun to bridge this gap by providing consistent evaluation and reproduction frameworks, we argue that dataset design can and should leverage the same principled methodology as model design. In particular, large-scale dataset construction can generally be decomposed into two phases: uncurated data collection and dataset filtering.

This challenge is especially pronounced in the emerging domain of Computer-Use agents. Graphical User Interfaces (GUIs) are central to how people engage with the digital world, serving as the primary medium for a wide range of daily activities. Large Language Models (LLMs) [32], originally developed as conversational chatbots, have proven capable of much more: their ability to comprehend complex language instructions and seamlessly integrate tools has revealed significant potential for automating complex tasks through Computer-Use agents [1, 13, 16, 56]. This progress has inspired the development of intelligent agents that can substantially streamline human workflows based on natural language instructions.

Early work in this area focused on language-only agents [12, 47, 55] that rely on closed-source, API-based LLMs such as GPT-4 [32], leveraging text-rich metadata like HTML inputs or accessibility trees to perform navigation and other tasks. However, this text-only paradigm faces fundamental limitations in practice: (1) unlike web environments, general Computer-Use scenarios do not always provide rich API-based metadata, and (2) users typically interact with interfaces visually—through screenshots—without access to the underlying structural information. These limitations underscore the need for Computer-Use visual agents that can perceive and interact with UIs in the same way humans do.

Given a user instruction, the Computer-Use task can be broadly decomposed into two components: (1) \textit{planning}—generating one or a sequence of actions to execute—and (2) \textit{grounding}—mapping the relevant UI element(s) to precise coordinates on the screen by autoregressively predicting the bounding box or - in our case - the center of the target UI element. Figure~\ref{fig:grounding_task} illustrates this grounding process: given a screenshot and a natural language instruction, a vision-language model predicts the target coordinates on the screen. This decomposition highlights the core difficulty of the problem: it demands both rich semantic understanding of UI elements and accurate pixel-level localization. To address these requirements, researchers have begun training vision-language models with these capabilities in mind. For instance, studies such as [11,15,17] utilize web screenshot datasets to enhance large multi-modal models' element-grounding abilities.

\begin{figure}[h]
\centering
\includegraphics[width=0.9\textwidth]{figures/GroundingTask.png}
\caption{Overview of the grounding task. Given a screenshot and a user instruction (e.g., ``Open a new tab.''), a Vision-Language Model (VLM) predicts the coordinates of the target UI element on the screen.}
\label{fig:grounding_task}
\end{figure}

Despite this progress, training multi-modal models for GUI visual agents continues to face several significant challenges:
\begin{enumerate}[label=(\alph*)]
    \item \textbf{Availability of Training Data:} Many state-of-the-art models remain closed-source, restricting access to their training data. Even among open-source models, the training data or its exact specification is often unavailable, making it difficult to reproduce or build upon existing work.
    \item \textbf{Training Data Composition:} Similar to the challenges faced by language-based Computer-Use agents, the availability of rich APIs for the web has led to an overrepresentation of web data in existing training sets. Professional application data, by contrast, is difficult to obtain with reliable annotations.
    \item \textbf{Resolution and Aspect Ratio Variability:} If Computer-Use agents are to be deployed in the real world rather than on standardized virtual machines, they must contend with widely varying screenshot aspect ratios, resolutions, and sizes. This necessitates either resizing screenshots—at the cost of information loss—or training models that can generalize across different resolutions and display configurations.
\end{enumerate}

\chapter{Related Work}
\input{chapters/1_Introduction/2_RelatedWork/VisionLanguageModels}
\subsection{Grounding Models and Computer-use Agents}

\textbf{GTA1}~\cite{gta1} introduces GUI test-time scaling agents and demonstrates the effectiveness of reinforcement learning for grounding tasks. GTA1 establishes strong baselines that our work builds upon and compares against.

\textbf{UI-TARS 1.5}~\cite{uitars} pioneers automated GUI interaction with native agents, providing base models suitable for further fine-tuning on grounding tasks.

\textbf{UI-Venus}~\cite{uivenus} builds high-performance UI agents with rejected sampling fine-tuning (RFT), demonstrating alternative training approaches for grounding models.

\textbf{SE-GUI}~\cite{segui} enhances visual grounding for GUI agents via self-evolutionary reinforcement learning, introducing data quality filtering methods that inform our approach.

\textbf{OmniParser}~\cite{omniparser} provides pure vision-based GUI agent capabilities and UI element detection, which we leverage in our filtering pipeline.


\section{GUI Grounding Datasets}

Several datasets have been developed for training GUI grounding models:

\textbf{ShowUI}~\cite{showui} presents a vision-language-action model for GUI visual agents, providing grounding data for both web and desktop interfaces. ShowUI demonstrates the importance of unified representations across different interface types.

\textbf{AutoGUI}~\cite{autogui} scales GUI grounding with automatic functional annotation, introducing methods to automatically generate grounding annotations from interface interactions.

\textbf{SeeClick}~\cite{seeclick} harnesses GUI grounding for advanced visual GUI agents, contributing early work on connecting natural language instructions to visual elements.

\textbf{PixMo Points}~\cite{pixmo}, part of the Molmo project, provides open weights and open data for state-of-the-art vision-language models, including pointing and grounding capabilities.

\textbf{OS-Atlas}~\cite{osatlas} introduces a foundation action model for generalist GUI agents, providing grounding data across diverse operating system interfaces.

\textbf{UGround}~\cite{uground} focuses on navigating the digital world as humans do through universal visual grounding for GUI agents.

\textbf{WaveUI}~\cite{waveui} contributes additional web and mobile interface grounding data.

\textbf{PC-Agent-E}~\cite{pcagente} demonstrates efficient agent training for computer use by extracting grounding annotations from recorded trajectories.

\textbf{UI-VISION}~\cite{uivision} provides a desktop-centric GUI benchmark for visual perception and interaction, with particular emphasis on professional applications.


\subsection{Evaluation Benchmarks}

\textbf{ScreenSpot-Pro}~\cite{li2025screenspotpro} offers GUI grounding evaluation for professional high-resolution computer use, providing a challenging benchmark for measuring grounding accuracy.

\textbf{OS-World-G}~\cite{xie2025jedi} scale computer-use grounding via user interface decomposition and synthesis, introducing both isolated grounding benchmarks and end-to-end agent evaluation tasks.


\section{Reinforcement Learning Methods}

\textbf{GRPO}~\cite{grpo} (Group Relative Policy Optimization), introduced in the DeepSeekMath work, provides an efficient reinforcement learning algorithm that we adapt for grounding model training.

\textbf{DAPO}~\cite{dapo} presents an open-source LLM reinforcement learning system at scale, demonstrating simplified RL objectives by removing KL-divergence terms, which we incorporate into our training approach.


\section{Contributions}
This thesis investigates data-centric and training strategies for GUI grounding tasks.
We unify multiple heterogeneous datasets into a single, standardized collection and conduct detailed ablations over different filtering methods.
We furthermore explore ways of enriching the resulting dataset in underrepresented domains such as professional desktop applications.
Finally, we study training dynamics ranging from supervised fine-tuning to reinforcement learning.
The main contributions are as follows:
\begin{enumerate}
    \item \textbf{Filtering pipeline.} We develop a comprehensive filtering pipeline that leverages a variety of pre-existing models to systematically curate a large, multi-source data pool into a high-quality training set.
    \item \textbf{Supplemental data collection.} We introduce a pipeline for collecting and annotating screen frames extracted from GUI tutorial videos, enabling the acquisition of training data for underrepresented domains that lack access to rich APIs typically available on the web.
    \item \textbf{Reinforcement learning training recipe.} We develop a reinforcement learning training recipe tailored to GUI grounding tasks and study the effects of task difficulty and reward function design on model performance.
    \item \textbf{State-of-the-art results.} We demonstrate that the resulting model, Gelato-30B-A3B, achieves 63.88\% accuracy on ScreenSpot-Pro and 69.15\%\,/\,74.65\% on OS-World-G\,/\,OS-World-G (Refined), surpassing prior specialized grounding models such as GTA1-32B and much larger vision-language models including Qwen3-VL-235B-A22B-Instruct.
    \item \textbf{Open source.} We publicly release the Gelato-30B-A3B model, the Click-100k dataset, and the accompanying code to foster further research and ensure reproducibility.
\end{enumerate}
\section{Thesis Structure}
The remainder of this thesis is organized as follows:
\begin{description}
    \item[Chapter 2 -- Methodology] Presents the data pool, followed by a data filtering pipeline, the collection of supplemental data to address gaps in the dataset, and the training of Gelato-30B-A3B.
    \item[Chapter 3 -- Results] Provides benchmarking performance on offline grounding benchmarks, as well as agentic performance on the OS-World benchmark.
    \item[Chapter 4 -- Conclusion \& Future Work] Summarizes the main contributions, discusses limitations of the current approach, and outlines promising future directions.
\end{description}



% Chapter 2: Methodology (including Data Curation and Training)
\chapter{Methodology}
\clearpage
\section{Data Pool}
\label{sec:data_curation}

\subsection{Data Sources}
We curate our training data from existing datasets for web and desktop GUI grounding. Each dataset sample contains a screenshot image, a natural language instruction describing the desired interaction, and ground truth bounding box coordinates for the target UI element. Our data pool draws from eight complementary sources: ShowUI \cite{lin2024showui}, AutoGUI \cite{li2025autogui}, PC-Agent-E \cite{he2025pcagente}, WaveUI \cite{agentsea2024waveui}, OS-Atlas \cite{wu2024osatlas}, UGround \cite{gou2024uground}, PixMo \cite{deitke2024molmo}, and SeeClick \cite{cheng2024seeclick}.

\paragraph{Normalization}
Since the collected datasets differ substantially in schema and annotation procedure, we apply a normalization step to consolidate them into a unified format. First, we partition samples by platform, discarding mobile samples. Second, because the source datasets span a spectrum from full interaction trajectories to isolated point annotations and encompass diverse action types (e.g., clicks, drags, text input), we extract and retain only click-action annotations pertinent to our grounding objective, discarding all other action types. The resulting annotation schema thus reduces to a triple of (image, instruction, bounding box). Table~\ref{tab:dataset_statistics} summarizes the resulting scale and composition of the normalized data pool, which comprises approximately 9.8 million training samples. Representative examples from six sources are shown in Figure~\ref{fig:data_source_samples} in the Appendix.

\begin{table}[h]
    \centering
    \caption{Dataset Statistics}
    \label{tab:dataset_statistics}
    \begin{tabular}{lrr}
    \hline
    \textbf{Dataset} & \textbf{Number of Samples} & \textbf{Number of Tokens (M)} \\
    \hline
    AutoGUI & 701,861 & 52.22 \\
    PC-Agent-E & 27,782 & 42.09 \\
    WaveUI & 24,977 & 1.74 \\
    SeeClick & 27,193 & 1.80 \\
    OS-ATLAS & 61,534 & 4.12 \\
    UGround & 8,290,455 & 618.48 \\
    ShowUI-Web & 598,856 & 40.48 \\
    ShowUI-Desktop & 7,496 & 0.48 \\
    PixMo Points & 92,477 & 5.81 \\
    \hline
    \textbf{Total} & \textbf{9,832,631} & \textbf{767.22} \\
    \hline
    \end{tabular}
    \end{table}

\subsection{Additional Processing}
Two datasets require additional preprocessing to extract suitable training samples. For PC-Agent-E, we extract individual click actions from recorded computer-use trajectories and generate corresponding natural language instructions by summarizing each action's reasoning chain with Claude 3.7 Sonnet. Since PixMo Points is a general-purpose grounding dataset not specifically designed for computer-use tasks, we employ Qwen2.5-7B-VL as a classifier to identify and retain only samples depicting valid computer screen images.

\paragraph{Resizing}
Since our work, as well as previous efforts, builds on top of the Qwen2.5-VL \cite{qwen2025qwen25vl} and Qwen3-VL \cite{qwen2025qwen3vl} model families, careful attention to architecture-specific preprocessing is required. Both models employ flexible patching mechanisms in which input images are dynamically resized to dimensions that are multiples of 28 pixels before being processed by the Vision Transformer (ViT). Failing to account for this resizing step can introduce coordinate misalignment between model predictions and ground truth annotations. To prevent such artifacts, we standardize all training and evaluation images by resizing them to the nearest valid dimensions prior to model input.

\paragraph{Resolution} To determine an appropriate image resolution for training, we conduct a preliminary study on the sensitivity of grounding performance to input resolution. Specifically, we evaluate GTA1 \cite{yang2025gta1} on two offline benchmarks---ScreenSpot Pro and ScreenSpot V2---while varying the maximum pixel budget from 1\,MP to 12\,MP. As shown in Figure~\ref{fig:resolution_impact}, the two benchmarks exhibit markedly different resolution dependencies. ScreenSpot Pro, which targets fine-grained elements in professional applications, benefits substantially from higher resolution: accuracy nearly doubles from 25.2\% at 1\,MP to a peak of 49.7\% at 4\,MP, before exhibiting diminishing returns at larger budgets. In contrast, ScreenSpot V2 already achieves strong performance at 1\,MP (89.5\%) and peaks at 2\,MP (91.8\%), with marginal degradation at higher resolutions. Based on these findings, we set the maximum resolution to 4\,MP for all training data, as this provides the best trade-off between grounding accuracy and computational cost.

\begin{figure}[h]
\centering
\includegraphics[width=0.85\textwidth]{figures/resolution_impact_benchmarks_gta1_7b.png}
\caption{Impact of maximum pixel budget on grounding accuracy for GTA1-7B. ScreenSpot Pro shows a strong positive correlation with resolution up to 4\,MP, while ScreenSpot V2 is largely resolution-invariant.}
\label{fig:resolution_impact}
\end{figure}

\paragraph{Coordinate System}
A further architectural distinction concerns coordinate representation. Qwen2.5-VL operates with absolute pixel coordinates, whereas Qwen3-VL adopts a normalized coordinate system scaled to the range $[0, 1000]$, which is reported to improve robustness to variations in image resolution and aspect ratio while simplifying post-processing. To maintain consistency with each model's native representation, we adapt the data preparation accordingly: absolute coordinates for Qwen2.5-VL experiments---facilitating direct comparison with other models built on the same backbone---and normalized coordinates for Qwen3-VL experiments.

\subsection{Data Quality Issues}
After assembling the data pool, we conduct a manual quality assessment by inspecting samples alongside their instructions and ground truth bounding boxes. As illustrated in Figure~\ref{fig:data_quality}, we identify three prevalent quality issues:
\begin{itemize}
    \item \textbf{Overly simple interactions}, such as ``Click on Health Conditions'', which can be trivially resolved through optical character recognition alone without requiring deeper semantic understanding of the GUI layout.
    \item \textbf{Misaligned annotations}, where the instruction text and target region diverge due to annotation errors in the source datasets.
    \item \textbf{Ambiguous tasks} that lack sufficient context for precise grounding or have multiple possible target regions.
\end{itemize}
We address these quality issues through a systematic filtering pipeline, described in Section~\ref{sec:data_filtering}.

\begin{figure}[h]
\centering
\includegraphics[width=\textwidth]{figures/DataQuality.png}
\caption{Examples of data quality issues encountered in curated datasets and their resolution through our filtering pipeline. Top left: misaligned bounding box; top right: simple task successfully filtered; bottom left: ambiguous task filtered by GTA1-7B; bottom right: unclear instruction filtered by GTA1-7B.}
\label{fig:data_quality}
\end{figure}

\clearpage
\section{Data Filtering}
\label{sec:data_filtering}

\subsection{Iterative Evaluation}
Developing an effective data recipe requires rapid empirical iteration. To enable this, we make two design choices that prioritize evaluation speed. First, we employ offline benchmarks---ScreenSpot-Pro, ScreenSpot V2, and OS-World---which can be evaluated in minutes, in contrast to online agentic benchmarks that require provisioning virtual machines and executing multi-step trajectories. Second, we use supervised fine-tuning (SFT) on Qwen2.5-VL rather than reinforcement learning, as SFT provides substantially faster training cycles. We operate under the assumption that data recipes that improve offline benchmark performance under SFT will transfer to online agentic settings and to RL-based training.

\subsection{Construction of Filtering Pipeline}
\paragraph{Balanced Weighting of Data Sources}
To ensure that the filtering pipeline is not biased towards any particular data source, we balance the data pool by sampling up to 50k samples from each source (or the maximum available if fewer than 50k exist).

\paragraph{OmniParser Filtering}
To address the misaligned bounding box annotations identified in Section~\ref{sec:data_curation}, we adopt a vision-based validation approach inspired by the IoU overlap strategy of \cite{yang2025gta1}. Specifically, we employ OmniParser \cite{lu2024omniparser}, a screen parsing system built on a YOLO-based detection model that identifies interactable UI elements by producing bounding boxes directly from screenshot pixels, without requiring access to underlying code structures or accessibility trees (Figure~\ref{fig:omniparser_yolo}). This purely visual approach enables robust parsing across diverse GUI environments---including web browsers and desktop applications---where traditional DOM-based methods are unavailable or unreliable.

We use OmniParser's detected bounding boxes to validate the spatial alignment between ground truth annotations and actual UI elements. Concretely, we discard any training sample whose ground truth click location falls outside all detected element bounding boxes, as illustrated in Figure~\ref{fig:data_quality}. This filtering step removes misaligned annotations that would otherwise introduce noise during training. Because the method operates purely on visual features, it generalizes across application types and platforms, making it particularly valuable for desktop-centric datasets where structured metadata is scarce.

\begin{figure}[h]
\centering
\includegraphics[width=0.9\textwidth]{figures/Omniparser.png}
\caption{OmniParser filtering pipeline. The system detects UI elements and their bounding boxes (shown as colored rectangles), enabling validation of ground truth annotations. Samples where the target click location falls outside all detected elements are filtered out as misaligned.}
\label{fig:omniparser_yolo}
\end{figure}


\paragraph{Difficulty Filtering}
Our difficulty filtering is inspired by \cite{yuan2025segui}, who filter out overly easy samples. However, their approach does not address overly difficult samples. We observe that ambiguous and unclear instructions---as identified among the data quality issues in Section~\ref{sec:data_curation}---are not captured by OmniParser alignment or easy-sample filtering alone, motivating an additional hard-sample filtering stage.

To filter samples by difficulty, we use either Qwen2.5-VL-7B or SE-3B \cite{yuan2025segui} for easy sample filtering, and GTA1-7B \cite{yang2025gta1}, UI-7B \cite{gu2025uivenus}, or SE-3B for hard sample filtering. Together with the OmniParser bounding box alignment step, these components form the complete filtering pipeline illustrated in Figure~\ref{fig:filtering_pipeline}. To validate this approach, we conduct detailed ablations by fine-tuning Qwen2.5-VL-7B on a source-balanced 10k subset and evaluating on ScreenSpot-Pro. We find that using Qwen2.5-VL-7B as the easy-sample filter and GTA1-7B as the hard-sample filter produces a +9 percentage point accuracy gain over unfiltered data and outperforms all other filtering model combinations. Furthermore, we find strong empirical evidence that hard-sample filtering is beneficial: adding GTA1-7B filtering on top of Qwen2.5-VL-7B filtering alone yields an additional +5.5 percentage point gain.

\begin{figure}[t]
\centering
\includegraphics[width=\textwidth]{figures/FilteringPipeline.png}
\caption{Overview of the complete data filtering pipeline. The raw data pool is processed through three filtering stages: OmniParser YOLO for bbox alignment filtering (removing samples where the ground truth does not overlap with any detected UI element), Qwen2.5VL-7B for easy sample filtering, and GTA1-7B for hard sample filtering. Only samples that pass all three stages are retained as high-quality training data.}
\label{fig:filtering_pipeline}
\end{figure}

\includegraphics[width=0.5\textwidth]{figures/FilteringAblation.png}
\includegraphics[width=0.5\textwidth]{figures/FilteringAllEvals.png}
While it might seem that filtering out hard samples with a stronger model like GTA1-7B would prevent the trained model from surpassing it, this is---counterintuitively---not the case. Previous work has shown that filtering a data pool using a weaker CLIP model can produce a stronger CLIP model when trained on the filtered data \cite{fang2023datafilteringnetworks}. One plausible explanation is that removing noisy or ambiguous samples enables the model to learn cleaner representations.


\subsection{Data Source Analysis}
In the previous filtering pipeline, we trained on a balanced set from all data sources. However, we are interested in identifying the performance of individual data sources. To investigate this, we trained Qwen 2.5 VL 7B on each data source separately, capping the number of samples at 4.9k to control for the confounding effect of data repetition.

\begin{table}[h]
\centering
\begin{tabular}{lccc}
\hline
Data Source & SS Pro & SS V2 & OS-World G \\
 & Accuracy & Accuracy & Accuracy \\
\hline
ShowUI-Web & \textbf{36.43\%} & 86.77\% & \textbf{48.24\%} \\
AutoGUI & 34.54\% & \textbf{87.68\%} & 47.06\% \\
PC-Agent-E & 34.28\% & 87.29\% & 47.84\% \\
WaveUI & 33.40\% & 87.16\% & 44.71\% \\
Omniact & 33.21\% & 87.16\% & 44.90\% \\
ShowUI-Desktop & 32.01\% & 85.21\% & 42.16\% \\
UGround & 31.82\% & 85.99\% & 41.76\% \\
Pixmo Points & 30.68\% & 86.64\% & 42.16\% \\
SeeClick & 29.22\% & 84.82\% & 43.53\% \\
\hline
\end{tabular}
\caption{Performance of individual data sources on downstream benchmarks. Each data source was trained separately on Qwen 2.5 VL 7B with 4.9k samples.}
\label{tab:individual_data_sources}
\end{table}
Table~\ref{tab:individual_data_sources} reveals substantial variation in data source quality across benchmarks. AutoGUI and PC-Agent-E consistently rank among the top sources on all three benchmarks, with ShowUI-Web also performing strongly on ScreenSpot-Pro and OS-World-G. In contrast, SeeClick and PixMo Points are consistently the weakest sources.

Overall, ShowUI-Web exhibits consistently strong performance across all benchmarks, particularly excelling on ScreenSpot Pro and OS-World G. PC-Agent-E demonstrates robust and balanced performance across all evaluation metrics, while AutoGUI achieves the highest ScreenSpot V2 accuracy alongside strong ScreenSpot Pro performance.

Conversely, SeeClick and Pixmo Points represent the poorest performing data sources, achieving the lowest (29.22\%) and second-lowest (30.68\%) ScreenSpot Pro accuracies, respectively.

Based on these data source experiments, we investigated the effect of removing the poorest performing data sources from the training pool. Specifically, we removed the bottom three data sources ranked by ScreenSpot Pro performance: SeeClick, Pixmo Points, and UGround. Subsequently, we trained the model on the remaining data sources in our filtered data pool sampling 10k samples.

\begin{table}[h]
\centering
\begin{tabular}{lcc}
\hline
Data Pool & SS Pro & SS V2 \\
 & Accuracy & Accuracy \\
\hline
All Data Sources & \textbf{45.22\%} & \textbf{91.05\%} \\
Remove Bottom 3 on SS Pro & 45.03\% & 90.14\% \\
(SeeClick, Pixmo Points, UGround) & & \\
Remove Worst one on SS Pro & 44.78\% & 90.79\% \\
(Pixmo Points) & & \\
\hline
\end{tabular}
\caption{Impact of removing poorest performing data sources on downstream task performance at equal data scale.}
\label{tab:data_source_removal}
\end{table}
As shown in Table~\ref{tab:data_source_removal}, removing the bottom three data sources at this stage does not yield consistent improvements across benchmarks. ScreenSpot Pro and ScreenSpot V2 accuracies decrease marginally. Given these mixed results, we conclude that it is beneficial to retain all data sources in the training pool to maximize overall performance. However, during RL training we find that these data sources contain problematic ambiguous annotations and ultimately exclude them from the RL data pool (Section~\ref{sec:dapo}).

\subsection{Data Sampling and Scale}
To investigate the impact of training set size on model performance, we conduct a series of ablation studies with varying data scales. From our filtered pool, we sample data from each source at different scales (10k, 20k, 35k, and 80k samples). As shown in Figure~\ref{fig:data_scaling}, we observe consistent performance improvements with increased data scale on ScreenSpot Pro, with accuracy improving from 45.22\% at 10k samples to 49.65\% at 80k samples when combined with enhanced prompting and additional in-house data. Performance on ScreenSpot V2 remains relatively stable across different scales, suggesting that model capacity and data quality play complementary roles in determining final performance.

\begin{figure}[h]
\centering
\includegraphics[width=0.85\textwidth]{figures/DataScaling.png}
\caption{Impact of training data scale on grounding accuracy. ScreenSpot Pro (left axis) shows consistent improvement with increasing data, while ScreenSpot V2 (right axis) remains relatively stable across scales.}
\label{fig:data_scaling}
\end{figure}

\subsection{Additional Experiments}
We conducted several additional experiments to optimize the training process, including training prompts, instruction rewriting, vision tower configuration, and learning rate tuning. Our hyperparameter search revealed that a learning rate of 1e-6 yields optimal performance, and that full fine-tuning (including the vision tower) significantly outperforms freezing the vision tower. We also found that the default training prompt is sufficient and that verbose tool-calling prompts do not improve performance. We experimented with rewriting existing instructions using LLM-based approaches (Qwen3-4B) as well as image-aware synthetic prompts (Qwen2.5-VL-7B) to improve instruction quality; however, this did not improve performance. Detailed results and ablations for these experiments are provided in the Appendix (see Tables~\ref{tab:prompt_ablation}, \ref{tab:prompt_rewriting}, \ref{tab:vision_tower}, and \ref{tab:learning_rate_sweep}).
\clearpage
\section{Supplemental Data}
After assembling the data pool (Section~\ref{sec:data_curation}) and filtering it (Section~\ref{sec:data_filtering}), we evaluate performance and identify systematic weaknesses to locate underrepresented areas in the training data.
We specifically focus on the ScreenSpot Pro benchmark, on which even closed-source models achieve poor performance. ScreenSpot Pro is dominated by professional applications, a domain that poses a significant gap for many GUI models because data collection is substantially more difficult due to the absence of structured APIs and the domain expertise required for annotation. Furthermore, the benchmark contains many high-resolution images, multi-window layouts, and complex screen configurations.

\subsection{Performance Analysis by Image Resolution and Aspect Ratio}
To better understand the characteristics of challenging samples in our evaluation set, we analyze model performance across different image resolutions and aspect ratios. We evaluate the 20k sample (without replacement) trained model on ScreenSpot Pro, stratifying performance by megapixel count and aspect ratio.

The analysis reveals three distinct observations:
\begin{itemize}
    \item The most common aspect ratio is 16:9 (approximately 1.78), and model performance decreases as we move toward ultra-wide aspect ratios (3.6).
    \item Performance declines with increasing image size, which we attribute to the resizing of images to a maximum of 4\,MP to control prompt length---this downsampling may remove fine-grained visual details necessary for accurate grounding.
    \item Surprisingly, model performance is also weak at low resolutions (2--3\,MP).
\end{itemize}

\begin{figure}[h]
\centering
\includegraphics[width=\textwidth]{figures/accuracy_by_resolution_and_aspect.png}
\caption{Model performance on the ScreenSpot Pro benchmark stratified by image resolution (left) and aspect ratio (right). The dashed red line indicates overall accuracy (46.5\%). Bar labels show per-category accuracy and sample count ($n$). The model achieves 736 correct predictions out of 1,581 total samples.}
\label{fig:resolution_aspect_performance}
\end{figure}

To mitigate these weaknesses, we explore data augmentation strategies.

\subsection{Data Augmentation}
We investigate two data augmentation strategies aimed at improving model performance on high-resolution screenshots, as evaluated on the ScreenSpot-Pro benchmark.

\paragraph{Composing High-Resolution Frames.}
To expose the model to high-resolution inputs during training, we experiment with synthetically composing dual-screen and large desktop montages. Specifically, we construct dual-screen samples with high aspect ratio by concatenating pairs of randomly selected frames from different data sources, comprising around 20\% of the augmented dataset. Additionally, we overlay random frames onto large desktop background images to simulate multi-window desktop environments. In this initial experiment, no verification is performed to ensure that instructions from different constituent frames do not spatially collide. As shown in Table~\ref{tab:composed_frames}, this naive composition strategy leads to a substantial degradation in ScreenSpot-Pro accuracy, decreasing from 45.22\% to 36.94\%.

\begin{table}[h]
\centering
\caption{Effect of composing high-resolution frames on ScreenSpot-Pro accuracy.}
\label{tab:composed_frames}
\begin{tabular}{lc}
\hline
\textbf{Configuration} & \textbf{SS-Pro Accuracy} \\
\hline
Baseline & 45.22\% \\
Composed High-Resolution Frames & 36.94\% \\
\hline
\end{tabular}
\end{table}

\paragraph{Random Image Upscaling.}
Motivated by findings from Phi-Ground~\cite{zhang2025phiground}, which reported an 8 percentage point improvement on ScreenSpot-Pro through random image upscaling during training, we evaluate a similar strategy. We randomly resize training images up to a maximum resolution of 4 megapixels. However, as shown in Table~\ref{tab:random_resize}, this approach does not yield improvements in our setting: ScreenSpot-Pro accuracy decreases slightly from 45.22\% to 44.14\%, while ScreenSpot~V2 accuracy drops more notably from 91.05\% to 88.45\%.

\begin{table}[h]
\centering
\caption{Effect of random image upscaling on benchmark accuracy.}
\label{tab:random_resize}
\begin{tabular}{lcc}
\hline
\textbf{Configuration} & \textbf{SS-Pro} & \textbf{SS V2} \\
 & \textbf{Accuracy} & \textbf{Accuracy} \\
\hline
Baseline & 45.22\% & 91.05\% \\
Random resize up to 4MP & 44.14\% & 88.45\% \\
\hline
\end{tabular}
\end{table}

These results suggest that naive augmentation strategies for high-resolution inputs---whether through frame composition or random upscaling---do not transfer effectively to our training setup and may introduce noise that degrades grounding performance.

\subsection{Professional Application Data}
\paragraph{UI Vision and JEDI}
To improve robustness on professional desktop applications which are underrepresented in comparison to web data, we incorporate UI-Vision \cite{nayak2025uivision} as additional training data. We do not use UI-Vision for evaluation, as it is not yet widely adopted as a standard benchmark for reporting final model performance. Instead, we treat it purely as supplementary supervision to increase coverage of desktop UI elements and interaction patterns.
UI-Vision provides bounding boxes for UI elements, semantic labels, and interaction-related metadata across a diverse set of professional and productivity software. Compared to web-based GUI datasets, UI-Vision focuses on complex desktop environments where structured APIs are typically unavailable and manual annotation requires domain knowledge. This makes it particularly valuable for improving model exposure to professional application layouts and visual element grounding.
The JEDI dataset \cite{xie2025jedi} includes large-scale grounding data sourced not only from web interfaces but also from real-world desktop and professional applications, incorporating screenshots and structured metadata from production software (e.g., office tools, system apps) and agent rollouts in environments like WindowsAgentArena.
We normalize, resize, and convert the UI-Vision \cite{nayak2025uivision} and JEDI \cite{xie2025jedi} datasets to the same format as the other datasets. Importantly, we do not apply the full filtering pipeline to these datasets, as the prior models used for difficulty filtering do not perform well in this domain either---applying such filtering would likely discard the majority of the collected data. After preprocessing, we obtain 5,733 samples from UI-Vision and 18,032 samples from JEDI.

To evaluate the impact of these supplemental sources, we fine-tune the model by incrementally adding UI-Vision and JEDI samples to the existing 38k-sample training set and compare against the baseline. Results are reported in Table~\ref{tab:supplemental_data_results}.

\begin{table}[h]
\centering
\caption{Impact of supplemental data on model performance. Adding UI-Vision and JEDI progressively improves OS-World-G while recovering ScreenSpot Pro accuracy at sufficient data scale.}
\label{tab:supplemental_data_results}
\small
\begin{tabular}{lcc}
\hline
\textbf{Model Configuration} & \textbf{SS Pro} & \textbf{OS-World-G} \\
\hline
SFT-7B (38k) & 49.3\% & 57.4\% \\
SFT-7B (38k) - 2 epochs & 50.16\% & 56.0\% \\
SFT-7B (44k) + UI-Vision & 47.6\% & 58.6\% \\
SFT-7B (49k) + UI-Vision + 5k JEDI & 47.9\% & 60.8\% \\
SFT-7B (63k) + UI-Vision + all JEDI & 50.09\% & 60.1\% \\
\hline
\end{tabular}
\end{table}

Adding UI-Vision alone initially decreases ScreenSpot Pro accuracy from 49.3\% to 47.6\%, while improving OS-World-G from 57.4\% to 58.6\%. As more JEDI data is incorporated, OS-World-G continues to improve, reaching 60.8\% with 5k JEDI samples. With the full JEDI set (63k total samples), ScreenSpot Pro recovers to 50.09\%---on par with the baseline---while OS-World-G remains substantially higher at 60.1\%. Notably, simply training the baseline for a second epoch yields only marginal gains on ScreenSpot Pro (50.16\%) and a decrease on OS-World-G (56.0\%), indicating that the improvements from supplemental data are not attributable to increased training steps alone.

\paragraph{Video Data Collection}
\label{sec:video_data_collection}
We additionally construct an automated pipeline for collecting and annotating screen frames from GUI tutorial videos to supplement our training data. The pipeline, illustrated in Figure~\ref{fig:video_annotation_pipeline}, comprises four stages: video acquisition, frame extraction, frame sampling, and instruction generation.

In the first stage, we manually curate a list of approximately 120 tutorial videos spanning over 80 professional applications, including 3ds~Max, ANSYS, Adobe Creative Suite, Blender, MATLAB, and Vivado, among others (see Table~\ref{tab:youtube_apps} in the Appendix for the full list). Videos are downloaded at their highest available quality.

In the second stage, we extract unique screen-like frames from each downloaded video using PySceneDetect. For each detected scene transition, the mid-frame is selected as a representative sample. We then apply a series of heuristic filters to remove non-screen content: face detection via OpenCV cascade classifiers excludes frames containing presenter overlays, while additional filters discard introductory and concluding segments, low-variance or solid-color frames, and grayscale frames. To eliminate near-duplicate content within each video, we perform perceptual deduplication using pHash.

In the third stage, we sample a fixed number of frames (default: 250) across a set of target professional applications from the extracted frame pool. The sampling procedure selects evenly spaced frames within each application to maximize temporal diversity.

In the fourth stage, we construct the final instruction dataset from the sampled frames. Natural language instructions are generated using Claude, producing structured instruction--target pairs for each screenshot (the full prompt is provided in Appendix~\ref{sec:appendix_claude_video_prompt}). Each training sample consists of a single screenshot paired with two instructions corresponding to distinct UI segments. 

\begin{figure}[t]
\centering
\includegraphics[width=\textwidth]{figures/ProfessionalAppData.png}
\caption{Examples of professional application screenshots collected via our video data pipeline, spanning Adobe Illustrator, AutoCAD, Stata, PyCharm, Vivado, and OriginPro. Each frame is paired with a natural language grounding instruction generated by Claude.}
\label{fig:professional_app_data}
\end{figure}

We incorporate the in-house professional application data collected from YouTube into the model-difficulty-filtered training set and sample 10k samples in total. As shown in Table~\ref{tab:youtube_professional_data}, this yields a modest improvement from 45.22\% to 46.11\% on ScreenSpot Pro.

\begin{figure}[t]
{\large\textbf{Video Data Collection and Annotation Pipeline}}\vspace{0.5em}
\centering
\includegraphics[width=\textwidth]{figures/VideoAnnotationPipeline.png}
\caption{Video data collection and annotation pipeline. Tutorial videos are processed through scene detection to extract representative frames, which are then filtered to remove non-screen content (e.g.\ presenter overlays, intro/outro segments, solid-color frames). A fixed number of frames per application is subsampled for temporal diversity, and finally Claude generates natural language grounding instructions paired with each screen frame.}
\label{fig:video_annotation_pipeline}
\end{figure}

\begin{table}[h]
\centering
\caption{Impact of In-house Professional Application Data}
\label{tab:youtube_professional_data}
\small
\begin{tabular}{lc}
\hline
\textbf{Data Configuration} & \textbf{SS Pro} \\
\hline
10k baseline & 45.22\% \\
10k + in-house prof. app data & 46.11\% \\
\hline
\end{tabular}
\end{table}
\clearpage


\section{Training}


\paragraph{Stabilizing RL Training}
Existing RL algorithms suffer from a gradient-decreasing problem when some prompts have accuracy equal to 1. For example, in GRPO, if all outputs $\{o_i\}_{i=1}^{G}$ of a particular prompt are correct and receive the same reward, the resulting advantage for this group is zero. A zero advantage results in zero policy gradients, shrinking the gradient magnitude and increasing the noise sensitivity of the batch gradient, thereby degrading sample efficiency.

Let the intended batch size be $B$ and the effective batch size be $B_{\text{eff}}$. Then
\[
B_{\text{eff}} = \sum_{i=1}^{B} \mathbf{1}\{A_i \neq 0\}.
\]

Under the i.i.d. assumption,
\[
B_{\text{eff}} \sim \mathrm{Binomial}(B, 1-p).
\]

The expected effective batch size is
\[
\mathbb{E}[B_{\text{eff}}] = B(1-p).
\]
The variance is
\[
\mathrm{Var}(B_{\text{eff}}) = B(1-p)p.
\]

This formulation reveals two problems: (1) a priori imbalance in sample difficulties, and (2) the number of samples with accuracy equal to 1 continues to increase during training, further exacerbating the problem. While DAPO focuses on online over-sampling and filtering to address this issue, we also employ a priori filtering. Without a priori filtering, we would need to oversample more aggressively and may exhaust the oversampling budget, resulting in increased variance in the effective batch size.

% TODO: does easyr1 implement DAPO style oversampling?

To this end, we propose filtering out prompts with accuracy equal to 1 and 0 before training, thereby decreasing the probability $p$ and maintaining more stable RL training throughout the optimization process.



% Chapter 3: Results
\chapter{Results}
\label{ch:results}

In this chapter, we present comprehensive evaluation results for Gelato-30B-A3B. We evaluate on isolated grounding benchmarks (ScreenSpot-Pro and OS-World-G) to measure grounding accuracy, and on the full OS-World agent benchmark to assess end-to-end agent performance. We compare against prior state-of-the-art models throughout.

\section{Grounding Benchmark Evaluation}
\label{sec:results_gelato30b}

\paragraph{Gelato vs.\ GTA1 Recipe.}
We first compare the Gelato training recipe to the GTA1 training recipe, since GTA1 is the prior state-of-the-art model for computer-use grounding. We initialize RL training from UI-TARS-1.5-7B and train on Click-100k until convergence (255 steps) using DAPO. As shown in Figure~\ref{fig:recipe_comparison}, Gelato outperforms GTA1-7B-2507 on both ScreenSpot-Pro and OS-World-G, with particularly large improvements on OS-World-G.

\begin{figure}[H]
    \centering
    \includegraphics[width=0.7\textwidth]{figures/FinalRecipeComparison.png}
    \caption{Comparison of training recipes applied to the same UI-TARS-1.5-7B base model. The Gelato recipe (DAPO on Click-100k) outperforms the GTA recipe on both ScreenSpot-Pro (+8.6~pp over base, +0.7~pp over GTA) and OS-World-G (+6.2~pp over base, +3.7~pp over GTA).}
    \label{fig:recipe_comparison}
\end{figure}

\medskip
\paragraph{Gelato vs.\ Other Models.}
Figure~\ref{fig:final_performance} places Gelato-30B-A3B in context against leading models, including the specialized grounding model GTA1 as well as end-to-end computer-use models (Qwen3-VL-30B-A3B-Instruct, Qwen2.5-VL-235B-A22B-Instruct, OpenCUA-72B). Despite using only 3B active parameters per token, Gelato-30B-A3B achieves \textbf{63.8\%} on ScreenSpot-Pro and \textbf{69.1\%} on OS-World-G, outperforming all other models on both benchmarks.

\begin{figure}[H]
\centering
\includegraphics[width=\textwidth]{figures/FinalPerformance.png}
\caption{Final benchmark comparison of Gelato-30B-A3B against leading models. \textbf{Left:} OS-World-G accuracy. \textbf{Right:} ScreenSpot-Pro accuracy. Gelato-30B-A3B achieves the highest accuracy on both benchmarks despite having fewer active parameters than all competitors.}
\label{fig:final_performance}
\end{figure}

\paragraph{Performance with Refusal.}
The OS-World-G benchmark includes a refusal subset where the correct answer is to decline grounding the instruction. We elicit refusal behavior from Gelato-30B-A3B without explicitly training for it. By appending ``If you cannot find the element, return refusal'' to the instruction prompt and including refusal cases in the evaluation (previously treated as zero-accuracy), we raise overall accuracy on OS-World-G to \textbf{69.15\%} (+1.96~pp) and on OS-World-G (Refined) to \textbf{74.65\%} (+1.25~pp).

\section{OS-World Agent Evaluation}

Having evaluated on isolated grounding benchmarks, we now assess Gelato-30B-A3B on the full OS-World agent benchmark, which measures end-to-end computer-use performance.

\subsection{Agent Harness}

Since Gelato-30B-A3B follows the two-stage paradigm (Section~\ref{par:end_to_end_vs_two_stage}), we use the GTA1.5 agent framework~\cite{osworld_gta15_agent} to evaluate it. The agent architecture combines:
\begin{itemize}
    \item \textbf{Planning model}: GPT-5 generates high-level action plans
    \item \textbf{Grounding model}: Gelato-30B-A3B grounds instructions to specific UI locations
    \item \textbf{Action execution}: The system executes clicks, keyboard input, and other actions
\end{itemize}
The agent has a maximum of 50 steps per task and waits 3 seconds between actions to allow the interface to update. We made minor modifications to the agent code, including a fix to the spreadsheet cell modification tool invocation and an additional delay between trajectory completion and evaluation to ensure the VM state fully updates.

\paragraph{Reproducibility Issues.}
We found that many of the issues discussed in the EpochAI article critiquing OS-World~\cite{brand2025osworld} significantly affect reproducibility. Benchmarking agent performance proved challenging due to:
\begin{enumerate}
    \item Non-deterministic planner behavior combined with insufficient trial repetitions, making fair comparison against prior work difficult.
    \item Changing evaluation prompts by OS-World authors without explicit versioning.
    \item Incomplete evaluation coverage and ambiguous task specifications that fail to recognize valid alternative solutions.
\end{enumerate}

\paragraph{Automated Evaluation Results.}
To enable fair comparison, we conducted three independent trials for both Gelato-30B-A3B and GTA1-32B in the same agent harness. All experiments were conducted on a fixed snapshot of OS-World to ensure reproducibility. Gelato-30B-A3B achieves \textbf{58.71 $\pm$ 0.66\%} success rate on OS-World automated evaluation, performing on par with or above GTA1-32B (\textbf{56.97 $\pm$ 1.47\%} success rate), thereby surpassing the previous state-of-the-art.

\begin{figure}[H]
\centering
\includegraphics[width=\textwidth]{figures/gelato-fig7.png}
\caption{OS-World agent performance across three runs with GPT-5 planner. Automated evaluation underestimates performance due to incomplete task specifications. Human evaluation shows Gelato-30B-A3B achieves \textbf{61.85\%} success rate vs.\ GTA1-32B's \textbf{59.47\%}.}
\label{fig:osworld_agent}
\end{figure}

\begin{table}[H]
    \centering
    \begin{tabular}{lccc}
    \hline
    \textbf{Model} & \textbf{Params} & \textbf{Architecture} & \textbf{OS-World (\%)} \\
    \hline
    Qwen2.5-VL-32B-Instruct & 32B & E2E & 8.83 \\
    UI-TARS-7B & 7B & E2E & 24.6 \\
    Qwen3-VL-30B-A3B-Instruct & 30B & E2E & 38.1 \\
    UI-TARS-1.5-7B & 7B & E2E & 42.5 \\
    OpenCUA-72B & 72B & E2E & 45.0 \\
    UI-TARS-2 & -- & E2E & 47.5 \\
    GTA1-32B + GPT-5 & 32B & 2S & 56.97 $\pm$ 1.47 \\
    \textbf{Gelato-30B-A3B + GPT-5} & \textbf{30B} & \textbf{2S} & \textbf{58.71 $\pm$ 0.66} \\
    \hline
    \end{tabular}
    \caption{Open-weights agent performance on OS-World (automated evaluation). \emph{Architecture}: End-to-End (E2E) models handle both planning and grounding; Two-Stage (2S) models use a separate planner (GPT-5) and grounding module.}
    \label{tab:osworld_agents}
\end{table}

\section{Human Evaluation}

Automated evaluation metrics can underestimate agent performance due to incomplete task specifications and ambiguous evaluation criteria. To obtain a more accurate estimate, we conduct human evaluation on a subset of tasks.

\subsection{Methodology}

We manually identified 20 tasks where the automated evaluation function is incomplete or the task specification is ambiguous. For each task, we review all agent trajectories across all runs and determine whether the task was successfully completed according to the intended goal.

Common issues with automated evaluation include:
\begin{itemize}
    \item Evaluation functions that check only one valid solution path when multiple valid approaches exist
    \item Timing issues where the evaluation runs before the final state is saved
    \item Over-specific checks that reject valid alternative solutions
\end{itemize}

\subsection{Human Evaluation Results}

With human evaluation corrections (Figure~\ref{fig:osworld_agent}), Gelato-30B-A3B achieves \textbf{61.85 $\pm$ 0.79\%} success rate compared to the automated result of \textbf{58.71 $\pm$ 0.66\%}. Similarly, GTA1-32B achieves \textbf{59.47 $\pm$ 1.27\%} with human evaluation compared to \textbf{56.97 $\pm$ 1.47\%} on automated evaluation. This amounts to a +3.14~pp gain for Gelato-30B-A3B and a +2.50~pp gain for GTA1-32B, confirming that automated evaluation significantly underestimates true agent performance. Importantly, the relative ranking between models remains consistent: Gelato-30B-A3B maintains its advantage over GTA1-32B under human evaluation.

\medskip
Note that this is not a comprehensive manual evaluation---we focused on clearly problematic tasks that were straightforward to verify. A more extensive human evaluation could reveal additional cases where automated metrics underestimate performance.

\section{Evaluation Summary}

Our evaluation demonstrates that Gelato-30B-A3B achieves state-of-the-art performance across multiple benchmarks:

\begin{enumerate}
    \item \textbf{Isolated grounding}: Surpasses specialized grounding models and much larger VLMs on both ScreenSpot-Pro and OS-World-G.
    \item \textbf{Superior data and training recipe}: The performance gains can be attributed to our curated Click-100k dataset and DAPO-based training recipe, as demonstrated by the controlled comparison against GTA1 on the same base model.
    \item \textbf{End-to-end agent performance}: Outperforms GTA1-32B and other end-to-end computer-use models on OS-World under both automated and human evaluation.
\end{enumerate}

These results validate our approach to data curation, filtering, and RL training, demonstrating that careful attention to data quality and training methodology yields significant improvements in grounding model capabilities.


% Chapter 4: Conclusion and Future Work
\chapter{Conclusion and Future Work}


\section{Conclusion}

This thesis presented Gelato-30B-A3B, a state-of-the-art grounding model for GUI computer-use tasks. Through careful data curation, model-based filtering, and reinforcement learning with DAPO, we achieved significant improvements over prior work on multiple benchmarks. Our key contributions are:

\begin{itemize}
    \item \textbf{Click-100k}: A high-quality, open-source grounding dataset built through principled filtering and enrichment of eight public data sources, supplemented with in-house professional application data.
    \item \textbf{Filtering methodology}: A model-based filtering pipeline that uses existing grounding models as difficulty and alignment judges, yielding a +9~pp accuracy gain over unfiltered data.
    \item \textbf{Training recipe}: An effective RL training recipe based on GRPO with DAPO-style dynamic sampling and asymmetric clipping, which consistently improves over SFT baselines.
    \item \textbf{State-of-the-art grounding}: 63.88\% on ScreenSpot-Pro and 69.15\%\,/\,74.65\% on OS-World-G\,/\,OS-World-G (Refined), surpassing both GTA1-32B and Qwen3-VL-235B-A22B-Instruct.
    \item \textbf{Agent performance}: 61.85\% success rate on OS-World (human evaluation), outperforming GTA1-32B in the same agent harness.
\end{itemize}

Grounding models remain a critical component of computer-use agents, whether in two-stage or end-to-end paradigms. As the field pushes toward more capable and general-purpose agents, continued improvements in grounding accuracy, efficiency, and robustness will be essential. By open-sourcing Click-100k, our trained models, and all evaluation artifacts, we aim to help close the gap between frontier and open-source efforts and accelerate progress in this area.

\section{Future Work}

While Gelato-30B-A3B advances the state of the art in GUI grounding, significant challenges remain on the path toward truly reliable computer-use agents. Even frontier models struggle to operate reliably across diverse applications and long-horizon tasks. We identify several promising directions for future work.

\paragraph{From Grounding to End-to-End Agents.}
Gelato operates as a grounding module within a two-stage pipeline and relies on a proprietary planning model (GPT-5). Training an open-weights end-to-end agent that jointly handles planning and grounding remains an important goal. A key bottleneck is the scarcity of high-quality trajectory data: collecting multi-step interaction traces across diverse applications is expensive and difficult to scale. As stronger computer-use models become available, they may serve as trajectory collectors, enabling open-source efforts to generate the training data needed for end-to-end agents.

\paragraph{Long-Horizon Reasoning.}
Current agents struggle with tasks that require long sequences of interdependent actions, particularly in professional applications where a single task may involve dozens of steps across multiple dialog boxes and menus. Improving reasoning capabilities on such long-horizon tasks---including hierarchical planning, efficient context management, and recovery from intermediate errors---is an important direction.

\paragraph{Trajectory-Level Reward Models.}
Our evaluation revealed significant difficulties with automated verifiers on the OS-World benchmark, where evaluation functions often fail to recognize valid alternative solutions. More broadly, reliable supervision of multi-step trajectories remains an open problem that must be addressed for future advances in trajectory-based data collection, evaluation, and RL training.

\paragraph{Professional Application Coverage.}
Despite our efforts to supplement training data with professional application screenshots (Section~\ref{sec:video_data_collection}), coverage of specialized software remains limited. Professional applications are inherently harder to collect data for and annotate, and models consequently perform worse in these domains. As stronger computer-use models become available, they could be leveraged to annotate additional professional application data at scale, which could then be open-sourced to benefit the broader research community.

% This displays the bibliography for all cited external documents. All references have to be defined in the file references.bib and can then be cited from within this document.
\bibliographystyle{IEEEtran}
\bibliography{references}

% Chapter 5: Appendix
\appendix
\chapter{Appendix}

\section{Data Source Examples}
\label{sec:appendix_data_sources}

Figure~\ref{fig:data_source_samples} shows representative examples from six of the data sources that comprise the Click-100k training set. Each sample consists of a screenshot paired with a natural language grounding instruction. The sources span web pages (ShowUI-Web, WaveUI), desktop applications across multiple operating systems (PC-Agent-E, OmniAct, ShowUI-Desktop), and mixed environments (OS-Atlas), illustrating the diversity of GUI contexts in the data pool.

\begin{figure}[h]
\centering
\includegraphics[width=\textwidth]{figures/data_source_samples.png}
\caption{Example training samples from six data sources in Click-100k. Each panel shows a screenshot with its associated grounding instruction (italic text below). The data pool covers web browsing, desktop applications, and professional software across Windows, macOS, and Linux environments.}
\label{fig:data_source_samples}
\end{figure}

\section{YouTube Tutorial Application List}

This section lists the professional applications covered by the YouTube tutorial data collection pipeline described in Section~\ref{sec:video_data_collection}.

\begin{table}[h]
\centering
\caption{Complete list of professional applications covered by the YouTube tutorial data collection pipeline.}
\label{tab:youtube_apps}
\begin{tabular}{llll}
\hline
3ds Max & Adobe Acrobat & Adobe After Effects & Adobe Dreamweaver \\
Adobe Illustrator & Adobe InDesign & Adobe Lightroom & Adobe Photoshop \\
Adobe Premiere & Airmail & Altium Designer & Android Studio \\
ANSYS & Apple Mail & Asana & Atom \\
AutoCAD & Autodesk Eagle & Autodesk Inventor & Autodesk Maya \\
Autodesk Revit & Avid Media Composer & Axure & Abaqus \\
Balsamiq & Blender & Brave & Burp Suite \\
Cadence Virtuoso & Catia & Cinema4D & COMSOL \\
Confluence & CryEngine & Cubase & DaVinci Resolve \\
DBeaver & Eviews & Figma & Final Cut Pro \\
FL Studio & Framer & Fusion 360 & GameMaker \\
GIMP & IBM SPSS & Intel Quartus Prime & IntelliJ IDEA \\
Jupyter & KiCad & LabVIEW & LibreOffice Base \\
LibreOffice Calc & LibreOffice Draw & LibreOffice Impress & LibreOffice Math \\
LibreOffice Writer & Logic Pro X & Looker & Mailbird \\
MATLAB & Microsoft Edge & ModelSim & Mozilla Firefox \\
Notion & OBS & Obsidian & OneNote \\
OriginLab & Outlook & Power BI & PyCharm \\
Quartus & RStudio & Simulink & SolidWorks \\
Stata & Tableau & Thunderbird & Unity \\
Unreal Engine & VLC Media Player & VMWare & Vivado \\
Xcode & & & \\
\hline
\end{tabular}
\end{table}

\section{Claude Prompt for Video Frame Annotation}
\label{sec:appendix_claude_video_prompt}

The following prompt was used with Claude to generate structured grounding instructions from professional application screenshots collected via the video data pipeline (Section~\ref{sec:video_data_collection}).

\begin{figure}[p]
\vspace*{-2em}
\begin{lstlisting}[
  basicstyle=\ttfamily\tiny,
  breaklines=true,
  frame=single,
  backgroundcolor=\color{black!5},
  xleftmargin=0.5em,
  xrightmargin=0.5em,
  aboveskip=0.3em,
  belowskip=0.3em,
  abovecaptionskip=0.5em,
  belowcaptionskip=0pt,
  caption={Prompt template used for Claude-based annotation of professional application screenshots.},
  label={lst:claude_video_prompt}
]
Analyze this user interface image and provide helpful
instructions for using this application.

First, identify the main segments/regions of the interface
(e.g., "top toolbar", "left sidebar", "main canvas",
"color palette popup", "file menu dropdown", etc.).

Then, generate 20-30 diverse, specific instructions
organized by these segments. Focus on:

1. Common tasks a user might want to accomplish
2. How to navigate and use the visible features
3. Workflow suggestions and tips
4. Specific actions related to the tools and functions shown

CRITICAL INSTRUCTION REQUIREMENTS:
- All instructions must be written in English (US), even
  if the interface uses another language
- ALL instructions must be LEFT CLICK based actions only
  (no right click, typing, keyboard shortcuts, drag/drop)
- Each instruction must be VERY exact and require only a
  SINGLE LEFT CLICK to accomplish
- Instructions must be concise: 1-15 words maximum
  (1 word if applicable)
- Be VERY specific in your instructions
- Reference specific UI elements and their purposes
- Provide actionable guidance that users can follow by
  left clicking
- Include both beginner and intermediate level left click
  actions
- Vary instruction structure: use diverse phrasings like
  "Open...", "Access...", "View...", "Save...",
  "Navigate to...", "Switch to...", "Add...", "Remove...",
  "Select...", etc.
- Avoid starting every instruction with "Click" - use
  natural action verbs that imply left clicking
- Focus on what's actually visible and left clickable in
  the interface
- Pay special attention to dynamic elements: open tabs,
  dropdowns, pop-ups, context menus, modal dialogs, and
  expandable sections if they exist
- Avoid overly verbose descriptions

Return your response as a JSON object with this exact
format:
{
  "interface_description": "Brief description of what
    type of interface this is and its main purpose",
  "segments": {
    "segment_name_1": {
      "description": "Brief description of this screen
        segment",
      "instructions": [
        "Specific instruction 1",
        "Specific instruction 2",
        ...
      ]
    },
    "segment_name_2": {
      "description": "Brief description of this screen
        segment",
      "instructions": [
        "Specific instruction 1",
        "Specific instruction 2",
        ...
      ]
    }
  }
}

DO NOT hypothesize features not visible in the image.
Only provide instructions based on what you can clearly
see.

REMEMBER: ALL instructions must be accomplished with a
SINGLE LEFT CLICK only. No right click, typing, keyboard
shortcuts, or multi-step actions.
Use diverse instruction phrasings - examples: "Open File
menu", "Save document", "View project structure",
"Access settings", "Navigate to homepage", "Switch tabs",
"Add new item", "Close dialog", "Expand dropdown",
"Select tab".
\end{lstlisting}
\end{figure}

\section{Instruction Relabeling Experiments}

\paragraph{Instruction Relabeling}
We want to understand how scaling the data impacts the SFT process. We trained Qwen 2.5 VL 7B on 10k samples with the hard samples filtered using GTA1 7B and easy samples filtered using Qwen 2.5 VL 7B. We also tried to improve the data by rewriting the prompts using LLM (Qwen3 4B) to remove noisy artifacts and by using image aware synthetic prompts (Qwen 2.5 VL 7B) to rewrite the prompts to include action intent.

\begin{table}[h]
\centering
\begin{tabular}{lcc}
\hline
Rewriting Strategy & SS Pro & SS V2 \\
 & Accuracy & Accuracy \\
\hline
No Rewriting & 45.22\% & 91.05\% \\
Rewritten Prompts using LLM & 42.25\% & 88.84\% \\
(Qwen3 4B) & & \\
Image-aware Synthetic Prompts & 39.27\% & 85.86\% \\
(Qwen 2.5 VL 7B) & & \\
\hline
\end{tabular}
\caption{Impact of prompt rewriting strategies on downstream task performance.}
\label{tab:prompt_rewriting}
\end{table}

The following prompt template was used with Qwen3 4B to clean up noisy UI instructions:

\begin{figure}[h]
\begin{lstlisting}[
  basicstyle=\ttfamily\small,
  breaklines=true,
  frame=single,
  backgroundcolor=\color{black!5},
  xleftmargin=1em,
  xrightmargin=1em,
  aboveskip=1em,
  belowskip=1em,
  caption={Prompt template used for instruction cleaning.},
  label={lst:cleaning_prompt}
]
You are a text editor that cleans up UI instructions.
Your task is to fix formatting, style, and noise issues
while preserving the exact intent and meaning.

Rules:
1. Fix grammatical errors and awkward phrasing
2. Remove overly verbose descriptions - keep instructions concise
3. Clean up technical noise (HTML tags, underscores, etc.)
4. NEVER change the core action or target element
5. Keep the output as a single, clear instruction

Examples:
Input:  "Choose Located in the top right corner."
Output: "Choose the element in the top right corner."

Input:  "Based on my descriptions, find the locations of
         the mentioned element in this webpage screenshot
         (with point). This element provides access to the
         main WhatsApp page, allowing users to navigate to
         the primary WhatsApp interface."
Output: "Find the element that provides access to the main
         WhatsApp page."

Input:  "click on new_tab"
Output: "Click on new tab."

Input:  "Click on the button labeled 'Submit' which is used
         to submit the form"
Output: "Click on the Submit button."

Now clean up this instruction:
{instruction}

Cleaned instruction:
\end{lstlisting}
\end{figure}

\section{Training Prompt Ablation}

Three system prompts were compared in this ablation. Their full text is shown below.

\begin{figure}[h]
\begin{lstlisting}[
  basicstyle=\ttfamily\small,
  breaklines=true,
  frame=single,
  backgroundcolor=\color{black!5},
  xleftmargin=1em,
  xrightmargin=1em,
  aboveskip=1em,
  belowskip=1em,
  caption={Default Prompt (no resolution).},
  label={lst:prompt_default}
]
You are an expert UI element locator. Given a GUI
image and a user's element description, provide the
coordinates of the specified element as a single
(x,y) point. For elements with area, return the
center point.
Output the coordinate pair exactly:
(x,y)
\end{lstlisting}
\end{figure}

\begin{figure}[h]
\begin{lstlisting}[
  basicstyle=\ttfamily\small,
  breaklines=true,
  frame=single,
  backgroundcolor=\color{black!5},
  xleftmargin=1em,
  xrightmargin=1em,
  aboveskip=1em,
  belowskip=1em,
  caption={Default Prompt + Image Resolution.},
  label={lst:prompt_default_res}
]
You are an expert UI element locator. Given a GUI
image and a user's element description, provide the
coordinates of the specified element as a single
(x,y) point. The image resolution is height {height}
and width {width}. For elements with area, return
the center point.
Output the coordinate pair exactly:
(x,y)
\end{lstlisting}
\end{figure}

\begin{figure}[h]
\begin{lstlisting}[
  basicstyle=\ttfamily\scriptsize,
  breaklines=true,
  frame=single,
  backgroundcolor=\color{black!5},
  xleftmargin=1em,
  xrightmargin=1em,
  aboveskip=1em,
  belowskip=1em,
  caption={Qwen Tool Calling Prompt + Image Resolution (abbreviated).},
  label={lst:prompt_qwen_tool}
]
You are a helpful assistant.

# Tools
You may call one or more functions to assist with the
user query. You are provided with function signatures
within <tools></tools> XML tags:

<tools>
{"type": "function", "function":
  {"name": "computer_use",
   "description": "Use a mouse and keyboard to interact
     with a computer, and take screenshots.
     * This is an interface to a desktop GUI. You do not
       have access to a terminal or applications menu.
     * Some applications may take time to start or process
       actions, so you may need to wait and take successive
       screenshots to see the results of your actions.
     * The screen's resolution is {width}x{height}.
     * Whenever you intend to move the cursor to click on
       an element, consult a screenshot to determine the
       coordinates of the element before moving the cursor.
     * Make sure to click any buttons, links, icons, etc
       with the cursor tip in the center of the element.",
   "parameters": {"properties":
     {"action": {"description": "The action to perform.",
       "enum": ["key", "type", "mouse_move", "left_click",
         "left_click_drag", "right_click", "middle_click",
         "double_click", "scroll", "wait", "terminate"],
       "type": "string"}, ...},
     "required": ["action"], "type": "object"}}}
</tools>

For each function call, return a json object with
function name and arguments within <tool_call></tool_call>
XML tags:
<tool_call>{"name": <function-name>,
  "arguments": <args-json-object>}</tool_call>
\end{lstlisting}
\end{figure}

We trained Qwen 2.5 VL 7B on 10k samples with different training prompts to investigate two questions: (1) whether the base Qwen model's verbose tool-calling computer-use prompt is necessary for good performance, and (2) the impact of including image resolution in the prompt.

We find that the default prompt is sufficient for the model to perform well, achieving competitive results across benchmarks. The impact of including image resolution in the prompt is mixed---while it slightly improves performance on ScreenSpot V2, it actually decreases performance on ScreenSpot Pro.

\begin{table}[h]
\centering
\begin{tabular}{lcc}
\hline
Prompt & SS Pro & SS V2 \\
 & Accuracy & Accuracy \\
\hline
Qwen Tool Calling Prompt + Image Resolution & 37.76\% & 88.59\% \\
Default Prompt + Image Resolution & 36.12\% & 88.59\% \\
Default Prompt & 39.03\% & 87.68\% \\
\hline
\end{tabular}
\caption{Impact of training prompt on downstream task performance for Qwen 2.5 VL 7B trained on 10k samples.}
\label{tab:prompt_ablation}
\end{table}

\section{Cold-Start SFT Budget}
\label{sec:coldstart}

Since \cite{qin2025uitars15} report a progressive training pipeline with continual SFT followed by RL, we conducted an experiment to understand the interaction between SFT and RL. We conducted cold-start experiments in which Qwen2.5-VL-7B-Instruct models are first fine-tuned on varying amounts of SFT data (1k, 3.3k, 10k, and 63k samples, each for a single epoch), and then trained with GRPO on the full 63k data pool for one epoch (63k was the size of our dataset at that time of the project). This experiment isolates how much the SFT initialization matters for subsequent RL gains.

\begin{figure}[h]
\centering
\includegraphics[width=\textwidth]{figures/rl_coldstart_scaling_overlay.png}
\caption{Cold-start SFT budget experiment. Models fine-tuned on varying amounts of SFT data (1k, 3.3k, 10k, 63k) are subsequently trained with GRPO. RL performance scales monotonically with SFT data budget, confirming that RL benefits from a stronger initialization.}
\label{fig:rl_coldstart}
\end{figure}

As shown in Figure~\ref{fig:rl_coldstart}, RL performance scales monotonically with SFT data budget, confirming that RL benefits from a stronger initialization and motivating our choice of the 63k SFT model as the starting point for subsequent experiments.

\section{RL Hyperparameter Ablations}
\label{sec:rl_ablations}

We ablate key RL hyperparameters---sampling temperature, learning rate, and KL penalty weight---to establish robust defaults for all subsequent experiments.

\paragraph{Temperature.}
Using the 10k SFT model, we compare temperatures of 0.65, 0.85, and 1.0. Temperature controls the diversity of rollouts and thus the exploration--exploitation trade-off during RL training.

\paragraph{Learning Rate and KL Weight.}
We compare our default hyperparameters (LR = $1 \times 10^{-6}$, KL coef = $1 \times 10^{-2}$) with parameters inspired by UI-Venus~\cite{gu2025uivenus} (LR = $4 \times 10^{-7}$, KL coef = $4 \times 10^{-3}$) on the 63k SFT model. Our results show robustness to moderate variations in these hyperparameters.

\section{Hyperparameter Search and Model Configuration}

\subsection{Vision Tower Fine-tuning}
We investigated whether to freeze or fine-tune the vision tower during training. Results show that full fine-tuning significantly outperforms freezing the vision tower.

\begin{table}[h]
\centering
\begin{tabular}{lc}
\hline
Model Configuration & ScreenSpot Pro Accuracy \\
\hline
Full fine-tune & 45.22\% \\
Freeze vision tower & 32.95\% \\
\hline
\end{tabular}
\caption{Comparison of full fine-tuning versus freezing the vision tower for Qwen 2.5 VL 7B.}
\label{tab:vision_tower}
\end{table}

\subsection{Learning Rate Search}
We performed a learning rate sweep for Qwen 2.5 VL 7B on the 10k model-filtered dataset using Qwen2.5VL-7B for easy sample filtering and GTA1-7B for incorrect sample filtering.

\begin{table}[h]
\centering
\begin{tabular}{lcc}
\hline
Learning Rate & ScreenSpot Pro & ScreenSpot V2 \\
 & Accuracy & Accuracy \\
\hline
1e-5 & 32.57\% & 80.02\% \\
5e-6 & 38.83\% & 87.80\% \\
\textbf{1e-6} & \textbf{45.35\%} & \textbf{90.27\%} \\
5e-7 & 38.07\% & 89.10\% \\
\hline
\end{tabular}
\caption{Learning rate sweep results. The optimal learning rate of 1e-6 (bold) achieves the best performance across all benchmarks.}
\label{tab:learning_rate_sweep}
\end{table}



\end{document}